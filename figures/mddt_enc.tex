% \documentclass{book}
% \usepackage{tikz, libertine}
% \begin{document}

\pgfmathsetmacro{\nodebasesize}{1} % A node with a value of one will have this diameter
\pgfmathsetmacro{\nodeinnersep}{0.1}

\tikzstyle{bloq} = [rectangle, draw, text badly centered, minimum height=2.5em, inner sep=1mm]

\tikzstyle{frame} =
[rectangle, fill=white, draw, text badly centered, minimum width=3*\h em,minimum height=3*\v em]

\tikzset{
	middlearrow/.style={
		decoration={markings,
		mark= at position 0.9 with {\arrow{#1}},
		},
		postaction={decorate}
	}
}

\newcommand{\bloq}[4]{% position, name options, label
	\node[bloq,#3] (#2) at (#1) {\textsf{#4}};
}
\newcommand{\add}[2]{%position, name
	\draw (4,4) circle [radius=0.3] node (add) {\textsf$+$};
}

\newcommand{\propnode}[5]{% position, name, options, value, label
	\pgfmathsetmacro{\minimalwidth}{sqrt (#4*\nodebasesize)}
	\node[#3,minimum width=\minimalwidth*1cm,inner sep=\nodeinnersep*0cm,circle,draw]
	(#2) at (#1) {#5};
}


\begin{tikzpicture}[scale=1.5,x=1em,y=1em]
\tikzstyle{every node}=[font=\small]
	\def\x{0}
	\def\y{0}
	\def\thickness{thick}
	\def\h{2.0}
	\def\v{1.0}

% INPUT VIDEO SEQUENCE
	\draw (\x-\h,\y+0.5*\v) node[above] {\textsf{Input Video Sequence}};
	\draw (\x-\h,\y-3*\v) node[above] {\textsf{Split into blocks}};
	\node[frame] at (\x-\h-0.3,\y-\v+0.3){};
	\node[frame] at (\x-\h-0.2,\y-\v+0.2){};
	\node[frame] at (\x-\h-0.1,\y-\v+0.1){};
	\node[frame] (sequence) at (\x-\h,\y-\v){};
	\draw[step=0.25,thin,help lines] (\x,\y) grid (-2*\h,-2*\v);
	\draw[step=0.5,thin] (\x,\y) grid (-2*\h,-2*\v);
	\draw[step=1.0,black,thick] (\x,\y) grid (-2*\h,-2*\v);

	\propnode{\x+1*\h,\y-1*\v}{add1}{fill=green!50}{0}{$+$}
	\propnode{\x+8*\h,\y-9*\v}{add2}{fill=green!50}{0}{$+$}
	\propnode{\x+0.75*\h/2,\y-1*\v}{inter1}{fill=black}{0.005}{}
	\propnode{2,\y-9*\v}{inter2}{fill=black}{0.005}{}

	\draw (\x+1.25*\h,\y-0.75*\v) node[above] {\textsf{residuals}};

	\bloq{\x+3*\h,\y-\v}{Transf}{text width=6em,fill=red!50}
	{$\cdots$\\[-0.25em]
	Transf. 10\\[-0.25em]
	$\cdots$\\[-0.25em]
	Transf. 18\\[-0.25em]
	$\cdots$\\[-0.25em]
	Transf. 26\\[-0.25em]
	$\cdots$\\[-1.5em]};
	\bloq{\x+6*\h,\y-\v}{Quant}{text width=6em,fill=yellow!50}{Quant.};
	\bloq{\x+8*\h,-\y-4*\v}{InvQuant}{text width=6em,fill=yellow!50}{Inv. Quant.};
	\bloq{\x+8*\h,\y-7*\v}{InvTransf}{text width=6em,fill=red!50}{Inv. Transf. N};
	\bloq{\x+8*\h,\y-12.5*\v}{Reconst}{text width=4em,fill=black!15}{Reconst.\\block};
	\bloq{\x+4.75*\h,\y-12.5*\v}{IntraPred}{text width=5em,text height=5em,fill=yellow!50}{};

	\draw[-latex,\thickness] (sequence) -- (add1) node [below right] {\textsf{$-$}};
	\draw[-latex,\thickness] (add1) -- (Transf);
	\draw[-latex,\thickness] (Transf) -- (Quant);
	\draw[-latex,\thickness] (Quant) -| (InvQuant);
	\draw[-latex,\thickness] (InvQuant) -- (InvTransf);
	\draw[-latex,\thickness] (InvTransf) -- (add2);
	\draw[-latex,\thickness] (add2) -- (Reconst);
	\draw[-latex,\thickness] (inter2) -- (add2);
	\draw[-latex,\thickness] (Reconst) -- (IntraPred);
	\draw[-latex,\thickness] (IntraPred) -| (add1);

	\def\startangle{45}
	\def\radius{0.707}
	\def\posx{\x+4*\h}
	\def\posy{\y-13*\v}

	\draw (\posx-3,\posy+1.5) node [below] {\footnotesize\textsf{Intra Prediction}};

	\tikzstyle{frame} =
	[rectangle, fill=white, draw, text centered, minimum width=4em, minimum height=2em]

	\filldraw[fill=black!15, draw=black!75] (\posx,\posy) rectangle (\posx+1,\posy+1);
	\filldraw[fill=black!15, draw=black!75] (\posx,\posy+1) rectangle (\posx+1,\posy+2);
	\filldraw[fill=black!15, draw=black!75] (\posx+1,\posy+1) rectangle (\posx+2,\posy+2);
	\filldraw[fill=white, draw=black!75] (\posx+1,\posy+1) rectangle (\posx+2,\posy);
	\filldraw[fill=black!15, draw=black!75] (\posx,\posy) rectangle (\posx+1,\posy-1);
	\filldraw[fill=black!15, draw=black!75] (\posx+2,\posy+2) rectangle (\posx+3,\posy+1);

	\draw[step=0.25,help lines] (\posx,\posy) grid (\posx+2,\posy+2);
	\draw[step=0.25,help lines] (\posx,\posy) grid (\posx+1,\posy-1);
	\draw[step=0.25,help lines] (\posx+2,\posy+2) grid (\posx+3,\posy+1);
	\draw[step=1,black!75,thick] (\posx,\posy) grid (\posx+2,\posy+2);
	\draw[step=1,black!75,thick] (\posx,\posy) grid (\posx+1,\posy-1);
	\draw[step=1,black!75,thick] (\posx+2,\posy+2) grid (\posx+3,\posy+1);

	\foreach \direction in {10,18,...,26}
	{%
		% set the angle of current prediction direction 
		\pgfmathsetmacro{\angle}{\startangle - (\direction - 2) * 180 / 32 + 180}
		\ifthenelse{\direction < 18}{\def\oper{cos}}{\def\oper{sin}}
		\pgfmathsetmacro{\radnew}{abs(\radius / \oper(\angle))}
		\ifthenelse{\direction=2 \OR \direction=10 \OR \direction=18 \OR \direction=26 \OR \direction=34}
		{
			\draw[opacity=0] (\posx+1.5,\posy+0.5) --++ (\angle:\radnew
			+ 0.25) node [opacity=1] {\footnotesize\textsf{\direction}};
		}{}
		\draw[middlearrow={latex reversed},very thin] (\posx+1.5,\posy+0.5) --++ (\angle:\radnew);
	}
	\draw(0.2,3) node {\includegraphics[width=3em]{./figures/resids-scan-vert}\footnotesize\textsf{10}};
	\draw(3.2,3) node {\includegraphics[width=3em]{./figures/resids-scan-diag}\footnotesize\textsf{18}};
	\draw(6.2,3) node {\includegraphics[width=3em]{./figures/resids-scan-horz}\footnotesize\textsf{26}};
	\draw[thick,round cap-<](2.71,0) --++ (0,1.5);
	\draw[thick,round cap-round cap] (-1,2) ++ ( 255 : 0 ) arc ( 255:285:15 );
\end{tikzpicture}
% \end{document}
